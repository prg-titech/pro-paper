\section{Introduction}

... similar opening as the other PWV papers before talking about PWV

\emph{Programming with versions} is a paradigm that introduces versions as part of the language semantics, managing them as a type's resources through a coeffectful type system~\cite{tanabe2021functional}. It was demonstrated in $\lambda$VL through versioned values that enables multi-version programming and allows for static dependency analysis by analyzing version resources in type aiming at increasing programming's flexibility and encouraging gradual dependency migration.

BatakJava~\cite{?}, an extension of Java, builds on the idea of $\lambda$VL and explores programming with versions in an object-oriented setting, treating versions as an element of class. It relaxes $\lambda$VL's limitation by allowing cross-version computations and includes versions in data structures and the module system.

However, BatakJava does not support \emph{version polymorphism}, which means a single definition cannot represent multiple implementations requiring different versions even when they are compatible. This inflexibility contradicts with the nature of multi-version programming that would require definitions to support instantiations by different versions.

We propose to add genericity to versioning in BatakJava through version parameterization, similar to type parameters in Java generics~\cite{}. Parameters would abstract version assignments to specific class instances in the definition.

Our contributions are as follows:

- We propose a version polymorphic extension of BatakJava that allows for version parameterization in class definitions
- We implement the extension by modifying BatakJava's type analysis and code generation
- We demonstrate the flexibility of programming with version polymorphism through a case study
